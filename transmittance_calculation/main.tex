\documentclass[11pt,addpoints,answers]{exam}
\usepackage[margin=1in]{geometry}
\usepackage{graphicx}
\usepackage[svgname]{xcolor}
\usepackage{url}
\usepackage{datetime}
\usepackage{color}
\usepackage[many]{tcolorbox}
\usepackage{hyperref}

\newcommand{\courseNum}{\href{https://learning3d.github.io/index.html}{16825}}
\newcommand{\courseName}{\href{https://learning3d.github.io/index.html}{Learning for 3D Vision}}
\newcommand{\courseSem}{\href{https://learning3d.github.io/index.html}{Spring 2024}}
\newcommand{\courseUrl}{\url{https://piazza.com/cmu/spring2024/16825}}
\newcommand{\hwNum}{Problem Set 3}
\newcommand{\hwTopic}{Volume Rendering}
\newcommand{\hwName}{\hwNum: \hwTopic}
\newcommand{\outDate}{Feb. 21, 2024}
\newcommand{\dueDate}{Mar. 13, 2024 11:59 PM}
\newcommand{\instructorName}{Shubham Tulsiani}
\newcommand{\taNames}{Anurag Ghosh, Ayush Jain, Bharath Raj, Ruihan Gao, Shun Iwase}

\lhead{\hwName}
\rhead{\courseNum}
\cfoot{\thepage{} of \numpages{}}

\title{\textsc{\hwName}} % Title


\author{}

\date{}


%%%%%%%%%%%%%%%%%%%%%%%%%%
% Document configuration %
%%%%%%%%%%%%%%%%%%%%%%%%%%

% Don't display a date in the title and remove the white space
% \predate{}
% \postdate{}
\date{}

%%%%%%%%%%%%%%%%%%
% Begin Document %
%%%%%%%%%%%%%%%%%%


\begin{document}

\section*{}
\begin{center}
    \textsc{\LARGE \hwNum} \\
    \vspace{1em}
    \textsc{\large \courseNum{} \courseName{} (\courseSem)} \\
    \courseUrl\\
    \vspace{1em}
    OUT: \outDate \\
    DUE: \dueDate \\
    Instructor: \instructorName \\
    TAs: \taNames
\end{center}


% Default to visible (but empty) solution box.
\newtcolorbox[]{studentsolution}[1][]{%
    breakable,
    enhanced,
    colback=white,
    title=Solution,
    #1
}

\begin{questions}
    \question \textbf{[10 pts]}
    \begin{figure}[h]
        \centering
        \includegraphics[width=\textwidth]{figure1.png}
        \caption{A ray through a non-homogeneous medium. The medium is composed of 3 segments ($y1y2$, $y2y3$, $y3y4$). Each segment has a different absorption coefficient, shown as $\sigma_1, \sigma_2, \sigma_3$ in the figure. The length of each segment is also annotated in the figure (1m means 1 meter).}
        \label{fig:q1}
    \end{figure}

    As shown in Figure~\ref{fig:q1}, we observe a ray going through a non-homogeneous medium.
    Please compute the following transmittance:
    \begin{itemize}
        \item $T(y1, y2)$
        \item  $T(y2, y4)$
        \item $T(x, y4)$
        \item $T(x, y3)$
    \end{itemize}

    \begin{tcolorbox}[fit,height=20cm, width=\textwidth, blank, borderline={0.5pt}{-2pt},halign=left, valign=center, nobeforeafter]
        \begin{studentsolution}
            According to slide \textbf{L09\_Volume\_Rendering Page 42}:
            \begin{align*}
                T(x, y) & = e^{-\sigma \|x - y\|} \\
                T(x, y) & = T(x, z) \cdot T(z, y)
            \end{align*}
            Thus everything's solved.
            \begin{align*}
                T(y_1, y_2) & = e^{-\sigma_1 \| y_1 - y_2 \|} = e^{-1 \times 2} = e^{-2}                                                                 \\
                T(y_2, y_4) & = e^{-\sigma_2 \| y_2 - y_3 \|} \cdot e^{-\sigma_3 \| y_3 - y_4 \|} = e^{-0.5 \times 1} \cdot e^{-10 \times 3} = e^{-30.5} \\
                T(x, y_4)   & = T(x, y_1) \cdot T(y_1, y_2) \cdot T(y_2, y_4) = 1 \times e^{-2} \times e^{-30.5} = e^{-32.5}                             \\
                T(x, y_3)   & = T(x, y_1) \cdot T(y_1, y_2) \cdot T(y_2, y_3) = 1 \times e^{-2} \times e^{-0.5} = e^{-2.5}
            \end{align*}
            Probably hitting an object between $y_3$ an $y_4$.
        \end{studentsolution}
    \end{tcolorbox}
\end{questions}
\end{document}
